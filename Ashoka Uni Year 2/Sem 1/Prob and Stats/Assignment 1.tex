\documentclass{article}

\title{P\&S Assignment 1}
\date{11/09/17}
\author{Divij Singh}

\begin{document}

	\maketitle
	
	\section{Q1}
	
		$\sum_{\omega\varepsilon\Omega}|P_1 (\omega_n) - P_2 (\omega_n)=|P_1 (\omega_1) - P_2 (\omega_1)| + |P_1 (\omega_2) - P_2 (\omega_2)| +...+ |P_1 (\omega_n) - P_2 (\omega_n)|$
		\\
		\\
		Let us take $x=P_1 (\omega_n)$ and $y=P_2(\omega_n)$
		\\
		\\
		Which means that we need to show that $|a-b| \le (a+b)$
		\\
		\\
		We can do this using proof by contradiction, i.e. let us prove that $|a-b|>(a+b)$
		There are two cases:
		\\
		\\
		(1)
			$a-b<a+b
			a-a>b+b
			2b<0$
			\\
			\\
			Which is a contradiction.
		\\
		\\
		(2)
			$a+b<a+b
			a+a<b-b
			2a<0$
			\\
			\\
			Which is a contradiction.
		\\
		\\
		Thus it must be that 
		$\sum_{\omega\varepsilon\Omega} |P_1 (\omega_1) - P_2 (\omega_1)| + |P_1 (\omega_2) - P_2 (\omega_2)| + |P_1 (\omega_n) - P_2 (\omega_n)|$
		\\
		\\
		Now, we can also write $\sum_{\omega\varepsilon\Omega}|P_1 (\omega_n) - P_2 (\omega_n)|$ as:
		$\sum P_1(\omega) + \sum P_2(\omega)
		= 1+1
		=2$
		\\
		\\
		Thus, we find that $\sum_{\omega\varepsilon\Omega}|P_1 (\omega_n) - P_2 (\omega_n)| \le 2$
		
	\section{Q2}
	
		We know that in order for 3 events, $A_1$,$A_2$ and $A_3$ to be independent, they must be 
		 \\
		 \\
		 (1) Independent in pairs
		 	$P(A_x \cap A_y) = P(A_x)P(A_y) for any x \ne y$
		 \\
		 \\	
		(2)And
			$P(A_1 \cap A_2 \cap A_3) = P(A_1)P(A_2)P(A_3)$
		\\
		\\	
		Now, if we make $\alpha = (A_1 \cap A_2), then P(A_1 \cap A_2 \cap A_3) = P(\alpha \cap A_3) = P(\alpha)P(A_3)$
		\\
		\\
		So, $\alpha$ and $A_3$ are independent.
		\\
		\\
		Now, for where the complement comes in, i.e. $\overline{A_1}$
		\\
		\\
		Using the proof we just showed, we know that the complement of an event is independent from the other two events.
		So we just have to show that (2) is also satisfied.
		\\
		\\
		So, let $\beta = A_2 \cap A_3$ , and let us partition it using $\overline{A_1}$
		$\beta = P(\beta \cap A_1) + P(\beta \cap \overline{A_1}$
		\\
		\\
		According to additivity of probability, this means that:
		$P\beta = P(\beta \cap A_1) + P(\beta \cap \overline{A_1}$
		\\
		\\
		i.e.
		\\
		$P(\beta \cap \overline{A_1}) = P(\beta) - P(\beta \cap A_1)$
		\\
		\\
		That is,
		$P( \overline{A_1} \cap A_2 \cap A_3) = P(A_2 \cap A_3) - P(A_2 \cap A_3)P(A_1) = P(A_2 \cap A_3)(1-P(A_1))$
		Hence $P(\overline{A_1} \cap A_2 \cap A_3) = P(A_2 \cap A_3)P(\overline{A_1}) = P(\overline{A_1})P(A_2)P(A_3)$
		
\end{document}
\documentclass{article}

\title{Algorithms 4}
\author{Divij Singh}
\date{15/02/18}

\usepackage{listings}
\lstset{
	language=Python
	}
\begin{document}

	\maketitle
	
	\section{}
	
	(a) 
		$2n^2+6=O(n^2)$\\
		As there exist positive constants, c and $n_0$\\
	(b) 
		$n^2 + 1 \neq O(n^3)$\\
		As the worst case complexity cannot be of a higher power than the highest power in the equation.\\
	(c) 
		$3n^2 + 5n +6 \neq O(n)$\\
		As the worst case power cannot be lower than the highest power in the equation.\\
	(d) 
		$2n + 5 =\Theta(n)$\\
		As positive constants $c_1$, $c_2$ and $n_0$ exist.\\
	(e) 
		$2n+5 \neq \Theta(n^2)$\\
		As the power of $\Theta(x)$ cannot be higher than that in the equation.\\
	(f) 
		$2n+5 \neq \Theta(1)$\\
		As there is no $n_0$ such that $0 \leq c_1f(n) \leq g(n) \leq c_2f(n)$\\
	(g) 
		$4n^2 + 3= \Omega(n^2)$\\
		As there is a point such that for all $n \geq n_0$, we have $0 \leq cf(n) \leq g(n)$\\
	(h) 
		$3n^2 +4=\Omega(n)$\\
		As there is a point such that for all $n \geq n_0$, we have $0 \leq cf(n) \leq g(n)$\\
	(i) 
		$5n^2+1 \neq \Omega(n^3)$\\
		As there is no point such that for all $n \geq n_0$, we have $0 \leq cf(n) \leq g(n)$\\
		
	\section{}
	In order for $g(n)=\Theta(f(n))$ to be true, it must be true that $f(n) = g(n)$\\
	However, in order for $g(n)= \Omega(f(n))$ to be true, $f(n)\leq g(n)$ must be true, and in order for $g(n)=O(f(n))$ to be true, $g(n)\leq f(n)$ must be true.\\
	Therefore it is not enough that just $g(n)= \Omega(f(n))$ and $g(n)=O(f(n))$ are true, but also that $g(n)=f(n)$\\
	
	\section{}
	
	As $b$ is positive, we know that $0\leq(1/2)^b n^b \leq (n+a)^b\leq 2^b n^b$\\
	Thus, with $c_1 = (1/2)^b$, $c_2=2^b$, and $n_0=2(a)$, we match the equation.
	
	\section{}
	The statement is meaningless, as Big-O notation is meant to give the worst-case runtime. Therefore, the statement should be that the running time of the algorithm is \textit{at most} $O(n^2)$
	
	\section{}
	This statement is accurate, as we saw in the proof of question number 2.
	
\end{document}
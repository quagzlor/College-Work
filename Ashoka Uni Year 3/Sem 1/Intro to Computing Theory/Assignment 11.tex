\documentclass{article}
\usepackage{amsmath}
\setcounter{MaxMatrixCols}{20}
\title{Introduction to Coding Theory Assignment 11}
\author{Divij Singh}
\date{05/03/19}


\begin{document}

	\maketitle
	
	\section{Q1}

We can do this by constructing a Hadamard matrix of order 8. This gives us a (8,8,4) linear code, or a [8,3,4] linear code. We drop the first column of this matrix, to get the required code:


\[
\begin{bmatrix}{0} & {0} & {0} & {0} & {0} & {0} & {0} \\ {1} & {0} & {1} & {0} & {1} & {0} & {1} \\ {0} & {1} & {1} & {0} & {0} & {1} & {1} \\ {1} & {1} & {0} & {0} & {1} & {1} & {0} \\ {0} & {0} & {0} & {1} & {1} & {1} & {1} \\ {1} & {0} & {1} & {1} & {0} & {1} & {0} \\ {0} & {1} & {1} & {1} & {1} & {0} & {0} \\ {1} & {1} & {0} & {1} & {0} & {0} & {1}\end{bmatrix}
\]\\\\\\\\\\\\\\\\\\
\\\\\\\\\\\\\\\
\section{Q2}
To have an [8,4,4] code, we must build a (8,16,4) linear code. For this we use a Hadamard matrix of order 8, and then add the complements of each row of the matrix to get the resulting code:
\[
\begin{bmatrix}
0&0&0&0&0&0&0&0\\
0&1&0&1&0&1&0&1\\
0&0&1&1&0&0&1&1\\
0&1&1&0&0&1&1&0\\
0&0&0&0&1&1&1&1\\
0&1&0&1&1&0&1&0\\
0&0&1&1&1&1&0&0\\
0&1&1&0&1&0&0&1\\
1&1&1&1&1&1&1&1\\
1&0&1&0&1&0&1&0\\
1&1&0&0&1&1&0&0\\
1&0&0&1&1&0&0&1\\
1&1&1&1&0&0&0&0\\
1&0&1&0&0&1&0&1\\
1&1&0&0&0&0&1&1\\
1&0&0&1&0&1&1&0\\
\end{bmatrix}
\]

\section{Q3}
We can do this by constructing a Hadamard matrix of order 12, and dropping the first column to get a code with the parameters (11,12,6)

\[
\begin{bmatrix}
0&0&0&0&0&0&0&0&0&0&0\\
1&0&1&0&0&0&1&1&1&0&1\\
1&1&0&1&0&0&0&1&1&1&0\\
0&1&1&0&1&0&0&0&1&1&1\\
1&0&1&1&0&1&0&0&0&1&1\\
1&1&0&1&1&0&1&0&0&0&1\\
1&1&1&0&1&1&0&1&0&0&0\\
0&1&1&1&0&1&1&0&1&0&0\\
0&0&1&1&1&0&1&1&0&1&0\\
0&0&0&1&1&1&0&1&1&0&1\\
1&0&0&0&1&1&1&0&1&1&0\\
0&1&0&0&0&1&1&1&0&1&1\\
\end{bmatrix}
\]

\section{Q4}
We can simply take the matrix from the previous question, and add the complement of each row to get the required code:
\[
\begin{bmatrix}
0&0&0&0&0&0&0&0&0&0&0\\
1&0&1&0&0&0&1&1&1&0&1\\
1&1&0&1&0&0&0&1&1&1&0\\
0&1&1&0&1&0&0&0&1&1&1\\
1&0&1&1&0&1&0&0&0&1&1\\
1&1&0&1&1&0&1&0&0&0&1\\
1&1&1&0&1&1&0&1&0&0&0\\
0&1&1&1&0&1&1&0&1&0&0\\
0&0&1&1&1&0&1&1&0&1&0\\
0&0&0&1&1&1&0&1&1&0&1\\
1&0&0&0&1&1&1&0&1&1&0\\
0&1&0&0&0&1&1&1&0&1&1\\
1&1&1&1&1&1&1&1&1&1&1\\
0&1&0&1&1&1&0&0&0&1&0\\
0&0&1&0&1&1&1&0&0&0&1\\
1&0&0&1&0&1&1&1&0&0&0\\
0&1&0&0&1&0&1&1&1&0&0\\
0&0&1&0&0&1&0&1&1&1&0\\
0&0&0&1&0&0&1&0&1&1&1\\
1&0&0&0&1&0&0&1&0&1&1\\
1&1&0&0&0&1&0&0&1&0&1\\
1&1&1&0&0&0&1&0&0&1&0\\
0&1&1&1&0&0&0&1&0&0&1\\
1&0&1&1&1&0&0&0&1&0&0\\
\end{bmatrix}
\]
\\\\\\\\\\\\\\\\\\\\\\\\\
\section{Question 5}
For this, we simply take a Hadamard matrix of order 12, and add the complement of each row.


\[
\begin{bmatrix}
0&0&0&0&0&0&0&0&0&0&0&0\\
0&1&0&1&0&0&0&1&1&1&0&1\\
0&1&1&0&1&0&0&0&1&1&1&0\\
0&0&1&1&0&1&0&0&0&1&1&1\\
0&1&0&1&1&0&1&0&0&0&1&1\\
0&1&1&0&1&1&0&1&0&0&0&1\\
0&1&1&1&0&1&1&0&1&0&0&0\\
0&0&1&1&1&0&1&1&0&1&0&0\\
0&0&0&1&1&1&0&1&1&0&1&0\\
0&0&0&0&1&1&1&0&1&1&0&1\\
0&1&0&0&0&1&1&1&0&1&1&0\\
0&0&1&0&0&0&1&1&1&0&1&1\\
1&1&1&1&1&1&1&1&1&1&1&1\\
1&0&1&0&1&1&1&0&0&0&1&0\\
1&0&0&1&0&1&1&1&0&0&0&1\\
1&1&0&0&1&0&1&1&1&0&0&0\\
1&0&1&0&0&1&0&1&1&1&0&0\\
1&0&0&1&0&0&1&0&1&1&1&0\\
1&0&0&0&1&0&0&1&0&1&1&1\\
1&1&0&0&0&1&0&0&1&0&1&1\\
1&1&1&0&0&0&1&0&0&1&0&1\\
1&1&1&1&0&0&0&1&0&0&1&0\\
1&0&1&1&1&0&0&0&1&0&0&1\\
1&1&0&1&1&1&0&0&0&1&0&0\\
\end{bmatrix}
\]

\end{document}

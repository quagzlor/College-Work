\documentclass{article}
\usepackage{amsmath}

\title{Introduction to Coding Theory Assignment 14}
\author{Divij Singh}
\date{08/03/19}


\begin{document}

	\maketitle
	
	\section{Q1}
	$e_R = 1,2$ , $e_L$ = does not exist
\section{Q2}
$e_L = 0,-1,1$ , $e_R$ = does not exist
\section{Q3}
$e_R =$  does not exist , $e_L =$ does not exist
\section{Q4}
(a)\\
$e_L = e_R$\\
Suppose both exist for $\circ$ over elements of S.\\
Then, look at $e_L \circ e_R$\\
This gives us $e_L \circ e_R = e_R$\\
But as $e_R$ is the right identity, we get $e_L \circ e_R = e_L$\\
As such, $e_L = e_R$\\\\
(b)
We already know that if $e_L$ and $e_R$ exist, then $e_L = e_R$\\
Let $e = e_L = e_R$, and have two distinct values $e_1$ and $e_2$, such that $e_1 \neq e_2$\\
Then we get $e_1 \circ e_2 = e_1$, and $e_1 \circ e_2 = e_2$\\
Therfore, either $e_L$ or $e_R$ is unique.
\section{Q5}
(a)\\\\
\(
\begin{bmatrix}
1/4 & 1/4\\
1/4 & 1/4\\
\end{bmatrix}
\)
\\\\
(b)\\\\
\(
\begin{bmatrix}
1/2 & 1/2\\
1/2 & 1/2\\
\end{bmatrix}
\)
\end{document}
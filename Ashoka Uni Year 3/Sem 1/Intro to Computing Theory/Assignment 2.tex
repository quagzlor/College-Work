\documentclass{article}

\title{Introduction to Coding Theory Assignment 2}
\author{Divij Singh}
\date{30/12/18}


\begin{document}

	\maketitle
	
	\section{Q1}
	(a)\\
$000 \rightarrow 010 p=0.028227$\\
$100 \rightarrow 010 p=0.000873$\\
$111 \rightarrow 010 p=0.000873$\\
Thus the decoded word is 000.\\
\\
(b)\\
$000 \rightarrow 011 p=0.000873$\\
$100 \rightarrow 011 p=0.000027$\\
$111 \rightarrow 011 p=0.028227$\\
Thus the decoded word is 111.\\
\\
(c)\\
$000 \rightarrow 001 p=0.028227$\\
$100 \rightarrow 001 p=0.000873$\\
$111 \rightarrow 001 p=0.000873$\\
Thus the decoded word is 000.\\
\\

\section{Q2}
(a)\\
$000 \rightarrow 010 p=0.147$\\
$100 \rightarrow 010 p=0.063$\\
$111 \rightarrow 010 p=0.063$\\
Thus the decoded word is 000.\\
\\
(b)\\
$000 \rightarrow 011 p=0.063$\\
$100 \rightarrow 011 p=0.027$\\
$111 \rightarrow 011 p=0.147$\\
Thus the decoded word is 111.\\
\\
(c)\\
$000 \rightarrow 001 p=0.147$\\
$100 \rightarrow 001 p=0.063$\\
$111 \rightarrow 001 p=0.063$\\
Thus the decoded word is 000.\\
\\

\section{Q3}
According to the maximum likeliehood rule:\\
$001 \rightarrow 000 p= 0.005$\\
$011 \rightarrow 000 p= 0.025$\\
The decoded word is 011\\
However, the nearest neighbour to 000 is 001.
\end{document}
\documentclass{article}
\usepackage{hyperref}

\title{Blockchain and Cryptocurrencies Final Exam}
\author{Divij Singh}
\date{10/05/19}


\begin{document}

	\maketitle
	
	\section{Q1}

The time would be 2019-05-06 03:26:56, according to \url{https://blockchain.com}

\section{Q2}
One such h would be 0, i.e the genesis block.

\section{Q3}
It would be roughly around 575,374, judging by trends on \url{https://blockexplorer.com}

\section{Q4}
Judging by statistic from blockchain.com, it would be around 410,849,309 total transactions.

\section{Q5}
Yes, it is possible. One account could be sending the currency to two or more adresses (not counting the address for change).

\section{Q6}
Yes, it is possible. The funds could come from a combination of accounts, rather than just one.

\section{Q8}
Currently, it takes roughly 9.8 minutes for a new block to be mined, according to blockchain.com. Thus, the nLockTime would be roughly 16 hours, 20 minutes from the time of txn creation.

\section{Q9}
Yes, as the two input scripts have a witness each. Though the ScriptSig field has information, it may be due to the sender using an older format.

\section{Q10}
Yes, as the input script has a witness. (Verified by checking on \url{https://yogh.io})

\section{Q11}
One such txn is 009f9991bbf6c99596d6cad637040630874a12968c381cec4ebecc11368654d5. It contains a Segwit input (contains a witness) and the output is a Segwit ID (it starts with 'bc1', which according to BIP-173 is a Segwit address type)\\
\url{https://github.com/bitcoin/bips/blob/master/bip-0173.mediawiki#Segwit_address_format}
\end{document}
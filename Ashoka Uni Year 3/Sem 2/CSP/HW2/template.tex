\documentclass{302}
\usepackage{amsmath}
\usepackage{seqsplit}

\author{Divij Singh}
\problem{2}
% \problem{A} means Problem Set A.
\collab{Aditya}
% or give names, e.g., \collab{Alyssa P. Hacker and A. Student}
\usepackage{xcolor}
\usepackage{pagecolor}

% comment the two lines below to have 
% black text white background
% But your submission should be black background 
% white text
\pagecolor{black}
\color{white}

\begin{document}

\section*{Problem 1-1}
(a) 20 words would have been consumed.\\
(b) $w_{32}$ to $w_{35}$ will be used.

\section*{Problem 1-2}
(a) 33 times\\
(b) 30 times\\
(c) 30 times\\
(d) 27 times

\section*{Problem 1-3}
(a) 11100110 (E6)\\
(b) 81ee66326c5078165c35b5a756cf120b\\
(c) There were 4 blocks of plaintext to convert (as there are 128 characters of hexadecimal) Additionally, one block of plaintext is repeated, as there are two similar blocks of ciphertext.

\section*{Problem 1-4}
PFA code

\section*{Problem 1-5}
The S-Boxes of DES were chosen specifically to prevent any access via a backdoor attack. Even a small change to the S-Boxes significantly weakens DES.

\section*{Problem 1-6}
The final step involves the rearranging of 32 bits from the S-boxes according to a given permutation. This is done so that the output bits from an S-box are spread across different S-boxes for the next round, as part of the confusion and diffusion concept.
 \section*{Problem 1-7}
PFA code and README
\end{document}

